\documentclass{article}
\usepackage[utf8]{inputenc}
\usepackage[spanish]{babel}
\usepackage{listings}
\usepackage{graphicx}
\graphicspath{ {images/} }
\usepackage{cite}

\begin{document}

\begin{titlepage}
    \begin{center}
        \vspace*{1cm}
            
        \Huge
        \textbf{Parcial 1 }
            
        \vspace{0.5cm}
        \LARGE
        Desafio
            
        \vspace{1.5cm}
            
        \textbf{Santiago Pereira Ramirez}
            
        \vfill
            
        \vspace{0.8cm}
            
        \Large
        Despartamento de Ingeniería Electrónica y Telecomunicaciones\\
        Universidad de Antioquia\\
        Medellín\\
        Marzo de 2021
            
    \end{center}
\end{titlepage}

\tableofcontents
\newpage
\section{Introduccion }\label{intro}
Se le indicara la forma de como desde una manera de como llevar unos objetos A a una posicion B.


Para el siguiente Desafio, sin ayuda y de manera individual, debera de realizarlo con las instrucciones mostradas a continuacion.Haga el ejercicio sobre una superficie plana como una mesa y sentado preferiblemente.

Tambien se le dira los elementos y la manera inicial en que debera poner los objetos para realizar el ejercicio planteado, Ademas es importante recalcar que despues de poner los objetos en el estado inicial, despues al comenzar a realizar las intrucciones estas desde la primera hasta la ultima se debera de hacer todo con una sola mano(preferiblemente con la dominante).


Cada vez que lea cada instruccion ejecutela inmediamente despues de leerla(si es necesario releer la instruccion hagalo cuantas veces sea).

\section{Elementos} \label{contenido}
Para la realizacion del ejercicio vamos a necesitar:

    -Una hoja de papel blanca
    
    -Dos tarjetas de tamaño similar(como cedula, TIP, Tarjeta bancaria,etc...)
    
    Estado inicial: ponga en la mesa las tarjetas una sobre la otralas dos tomando las misma direccion, y ponga la hoja de papel encima de las tarjetas sin que la hoja se salga de la superficie plana o mesa.

\section{Instrucciones} \label{contenido}
    ¡NOTA! Recuerde que apartir de aca el Desafio se debe de hacer con una sola mano(Preferiblemente con la dominate).
    
    
    -Coger la hoja, dezplazarla hacia un lugar cerca de las tarjetas sobre la superficie que estamos realizando el ejercicio y ademas asegurandonos de que la hoja o partes de la misma no toquen las tarjetas, a su vez tambien que la hoja de papel quede totalmente sobre la superficie plana(evitando la caida de esta).

    
    -coger y sostener las tarjetas garantizando que estas queden en direccion vertical en su mano.

    
    -Movera su mano(con la cual esta sosteniendo las tarjetas)hacia un punto cerca del centro de la hoja de papel  y reposara las bases de las tarjetas en la hoja, todavia    usted sosteniendolas en su mano(en direccion vertical).
 
    
    -Coloque su dedo indice sobre la cara exterior de una de las tarjetas y con su dedo medio o corazon tambien coloquelo sobre la cara exterior de la otra tarjeta (los dedos debe de estar cerca del top,cima  o parte superior de las tarjetas).

    
    -Haga presion sobre las caras exteriores de las tarjetas  con sus dedos(dedos indicados en el paso anterior )y asi las caras interiores de las tarjetas en la parte superior de estas estaran adheridos por la presion hecha por los dedos.
   
    
    -Con su dedo pulgar, con mucho cuidado desde un poco mas de la mitad hacia abajo de las tarjetas, separe las tarjetas sin desjuntar o dividir la parte superior(adderiendolas como se indica en los anteriores pasos).
    
    
    -Ahora    trate de equilibrar las tarjetas de modo de que se sostengan en forma de triangulo equilatero, cono o la torre Eiffel.

\end{document}
