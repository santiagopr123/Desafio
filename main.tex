\documentclass{article}
\usepackage[utf8]{inputenc}
\usepackage[spanish]{babel}
\usepackage{listings}
\usepackage{graphicx}
\graphicspath{ {images/} }
\usepackage{cite}

\begin{document}

\begin{titlepage}
    \begin{center}
        \vspace*{1cm}
            
        \Huge
        \textbf{Parcial 1 }
            
        \vspace{0.5cm}
        \LARGE
        Desafio
            
        \vspace{1.5cm}
            
        \textbf{Santiago Pereira Ramirez}
            
        \vfill
            
        \vspace{0.8cm}
            
        \Large
        Despartamento de Ingeniería Electrónica y Telecomunicaciones\\
        Universidad de Antioquia\\
        Medellín\\
        Marzo de 2021
            
    \end{center}
\end{titlepage}

\tableofcontents
\newpage
\section{Introduccion }\label{intro}
Para el siguiente Desafio, sin ayuda y de manera individual, debera de realizarlo con las instrucciones mostradas a continuacion.Haga el ejercicio sobre una superficie plana como una mesa y sentado preferiblemente.

\section{Elementos} \label{contenido}
Para la realizacion del ejercicio vamos a necesitar:

    -Una hoja de papel blanca
    
    -Dos tarjetas de tamaño similar(como cedula, TIP, Tarjeta bancaria,etc...)

\section{Instrucciones} \label{contenido}
    
    -

\bibliographystyle{IEEEtran}
\bibliography{references}

\end{document}
